\documentclass[12pt]{book} 
\usepackage[utf8]{inputenc} 
\usepackage[T1]{fontenc}
\usepackage[slovene]{babel} 	
\usepackage{amsmath} 
\usepackage{amssymb} 
\usepackage{amsthm}
\usepackage{lmodern}
\usepackage{graphicx}
\usepackage{biblatex}           
\usepackage{hyperref}   
\usepackage{geometry}

\geometry{
    a4paper,
    left=30mm,
    right=30mm,
    top=30mm,
    bottom=40mm
}

\makeatletter
\renewcommand{\maketitle}{
  \begin{titlepage}
    \begin{center}
      \vspace*{25mm} 
      \Huge\@title\par 
      \vspace{20mm} 
      \large\@author \\
      \vspace{140mm} 
      \large\@date\par 
    \end{center}
  \end{titlepage}
}
\makeatother

\usepackage{fancyhdr}
\pagestyle{fancy}
\fancyhf{}
\fancyhead[R]{\leftmark}
\renewcommand{\headrulewidth}{0.4pt} 
\linespread{1.3}

\usepackage{titlesec}
\titleformat{\chapter}[display]{\normalfont\Huge\bfseries}{\chaptertitlename\ \thechapter}{20pt}{\Huge}
\titleformat{\section}{\LARGE\bfseries}{\thesection}{1em}{}


\def\N{\mathbb{N}}
\def\R{\mathbb{R}}
\def\n{\noindent}
\def\s{\vspace{10pt}}


\theoremstyle{definition}
\newtheorem{definicija}{Definicija}

\theoremstyle{plain}
\newtheorem{izrek}{Izrek}

\theoremstyle{plain}
\newtheorem{trditev}{Trditev}

\theoremstyle{plain}
\newtheorem{posledica}{Posledica}

\theoremstyle{remark}
\newtheorem*{opomba}{Opomba}

\usepackage{thmtools}
\declaretheoremstyle[
    spaceabove=10pt,
    spacebelow=10pt,
    bodyfont=\normalfont,
    headfont=\bfseries,
    postheadspace=0.5em,
    qed=$\lozenge$ 
]{example}
\declaretheorem[style=example, unnumbered]{zgled}


\usepackage{filemod}
\title{\Huge Verjetnost 1}
\author{Napisal: Jon Pascal Miklavčič}
\date{\filemodprintdate{\jobname}}


\begin{document}

\frontmatter

\maketitle

\chapter*{Predgovor}
Zapiski v tej skripti so bili osnovanih na rokopisih predavanj iz študijskih let pred letom 2023 in doknčno dopolnjeni z zapiski iz predavanj iz leta 2024.

\tableofcontents

\mainmatter

\chapter{Neformalni uvod v verjetnost}

Začetki verjetnosti so v 17. stoletju, iz iger na srečo (kartanje, kockanje, \dots): 
\begin{itemize}
    \item 17. stol.: Fermant, Pascal, Bernulli;
    \item 18./19. stol.: Laplace, Poisson, Čebišev, Markov;
    \item 20. stol.: Kolmogorov.
\end{itemize}

\n Izvajamo poskus in opazujemo določen pojav, ki ga imenujemo \emph{dogodek}. Ta se lahko zgodi ali ne. 

\begin{zgled}
    Poskus je met kocke. Da pade šestica, da pade sodo število pik pa sta dogodka.
\end{zgled}

\n Poskus ponovimo $n$-krat. Opazujemo dogodek $A$. S $k_n(A)$ označimo \emph{frekvenco dogodka} $A$, t.j. število tistih ponovitev poskusa, pri katerih se je dogodek $A$ zgodil. Naj bo $f_n(A)=\frac{k_n(A)}{n}$ \emph{relativna frekvenca} dogodka $A$. Dokazati je mogoče, da zporedje $\left\{f_n(A)\right\}_n$ konvergira k nekemu številu $p \in [0, 1]$; $f_n(A) \xrightarrow{n \to \infty} p$. Dobimo:

\textbf{Statistično definicijo verjetnosti:}

$$P(A):= p $$

\n Pogosto lahko verjetnost določimo vnaprej in sicer s: 

\textbf{Klasično definicijo verjetnosti:}

$$P(A):= \frac{\# \text{ ugodnih izidov za dogodek } A}{\# \text{ vseh izidov}}$$

pri pogoju, da imajo vsi izidi \emph{enake} možnosti.

\begin{zgled}
    Met poštene kocke: 
    $$P(\text{sodo število pik}) = \frac{3}{6} = \frac{1}{2}$$
\end{zgled}

\begin{zgled}
    Kolikšna je verjetnost, da pri metu dveh poštenih kock znaša vsota pik $7$?
    
    \n Možne vsote so: $2, 3, 4, \dots , 12$. Opazimo, da je vseh vsot $11$ in od tega $1$ ugodna. Ali to pomeni, da $P(A)=\frac{1}{11}$. Ne! Izidi niso enkaoverjetni. 
    
    Na primer $2$ lahko dobimo samo kot $2=1+1$, $5$ pa kot $5=2+3=1+4=4+1=3+2$. 

    Torej vsi možni izidi, bodo urejeni pari $(x, y)$, kjer $x, y \in[1,6] \subset \mathbb{N}$
    $$
    \begin{array}{cccc}
        (1,1) & (1,2) & \cdots & (1,6) \\
        (2,1) & (2,2) & \ddots & (2,6)\\
        \vdots & \ddots & \ddots & \vdots \\
        (6,1) & (6,2) & \cdots & (6,6) \\
    \end{array}
    $$

    Vseh izidov je torej $36$ in od tega je $6$ ugodnih. Torej $P(A)= \frac{6}{36}=\frac{1}{6}$.
\end{zgled}

\n Če je izidov neskončno, si lahko pomagamo s \textbf{Geometrijsko definicijo verjetnosti.}

\begin{zgled}
    Osebi se dogovorita za srečanje med 10. in 11. uro. Čas prihoda je slučajen. Vsak od njiju po prihodu čaka največ 20 minut. Če v tem času drugega ni, odide. Najdlje čaka do 11. ure. Kolišna je vrejetnost srečanja?

    Čas začnemo šteti ob 10. uri. Vsi izidi so urejeni pari $(x, y) \in [0,1] \times[0,1]$. Ugodni izidi so $|x-y| \leq \frac{1}{3}$. Torej: 
    $$
    \begin{aligned}
        \text{1) } x \geq y &: x-\frac{1}{3} \leq y \\
        \text{2) } x \leq y &: y-x \leq \frac{1}{3} \iff y \leq x+\frac{1}{3}
    \end{aligned}
    $$

    Torej je $$P(\text{srečanja})=\frac{1-\left(\frac{2}{3}\right)^2}{1}=\frac{5}{9}$$
\end{zgled}

\n Teorija mere se ukvarja z splošnim zapisom geometrijske definicije. 

\begin{zgled}
    Vzamemo $m, n \in \mathbb{N}$, $m>n$. $n$ krogljic slučajno razporedimo v $m$ posod. Kolikšna je verjetnost dogodka, da so vse krogljice v prvih $n$ posodah, v vsaki ena?

    To je pomankljivo zastavljena naloga. Ne vemo namreč, ali med seboj krogljice razlikujemo, ali ne. Za dodatno predpostavko se ponujajo 3 možnosti: 

    1) \textbf{krogljice razlikujemo:}
    
    \n Število vseh izidov v tem primeru je ravno število \emph{variacij} $m$ elementov na $n$ mestih \emph{s ponavljanjem}. Za vsako od $n$-tih kroglic imamo $m$ možnosti, torej je vseh možnosti $m \cdot m \cdots m=m^n$.
    
    \n Število ugodnih izidov pa je ravno število \emph{permutacij} $n$ krogljic v prvih $n$ posodah. Torej je ugodnih možnosti $n(n-1) \ldots 2 \cdot 1=n!$.
    
    Torej je $$P(A)=\frac{n !}{m^n}$$

    2) \textbf{krogljic ne razlikujemo:} 
    
    \n V vsaki posodi je lahko več krogljic. Število vseh izidov je ravno število \emph{kombinacij s ponavljanjem}. Število kombinacij $m$ elementov s ponavljanjem na $n$ mestih je: 
    $$
    \binom{n+m-1}{n} = \binom{n+m-1}{m-1}
    $$

    \n Postavimo $n$ krogljic in med njih razporedimo $m-1$ črtic, ki predstavljajo stene posod:
    $$
    |\underbrace{\circ|\circ|\circ\circ |\circ\circ  \cdots |\circ |}_{n \text{ krogljic, } m-1 \text{ črtic}}|
    $$

    \n Na $n+m-1$ mestih moramo določiti $n$ krogljic. Ugoden izid je samo eden: 
    $$
    |\circ|\circ|\circ|\circ \cdots \circ|||| \cdots |
    $$

    Torej je $$P(A)=\frac{1}{\binom{n+m-1}{n}}$$

    3) \textbf{krogljic ne razlikujemo, v vsaki posodi je kvečjemu ena krogljica:} 
    
    Število vseh izidov je ravno število \emph{kombinacij brez ponavljanja} $\binom{m}{n}$. Ugoden izid je eden. 

    Torej je $$P(A)=\frac{1}{\binom{m}{n}}$$
\end{zgled}

\begin{opomba}
V fiziki so krogljice delici (atomi, molekule, ...), posode pa fazna stanja, v katerih so lahko delci. Glede na zgornje primere ločimo:

\begin{enumerate}
    \item Maxwell-Boltzmannovo statistika, ki velja za molekule plina.
    \item Bose-Einsteinovo statistika, ki velja za delce imenovane bozoni.
    \item Fermi-Diracovo statistika, ki velja za fermione. 
\end{enumerate}

\n Diracovo izključitveno načelo.

\end{opomba}

\newpage

\chapter{Aksiomatična definicija verjetnosti}


\end{document}