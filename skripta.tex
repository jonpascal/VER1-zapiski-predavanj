\documentclass[12pt]{book} 
\usepackage[utf8]{inputenc} 
\usepackage[T1]{fontenc}
\usepackage[slovene]{babel} 	
\usepackage{amsmath} 
\usepackage{amssymb} 
\usepackage{amsthm}
\usepackage{lmodern}
\usepackage{graphicx}
\usepackage{enumitem}
\usepackage{biblatex}           
\usepackage{hyperref}   
\usepackage{geometry}

\geometry{
    a4paper,
    left=30mm,
    right=30mm,
    top=30mm,
    bottom=40mm
}

\makeatletter
\renewcommand{\maketitle}{
  \begin{titlepage}
    \begin{center}
      \vspace*{25mm} 
      \Huge\@title\par 
      \vspace{20mm} 
      \large\@author \\
      \vspace{140mm} 
      \large\@date\par 
    \end{center}
  \end{titlepage}
}
\makeatother

\usepackage{fancyhdr}
\pagestyle{fancy}
\fancyhf{}
\fancyhead[R]{\leftmark}
\renewcommand{\headrulewidth}{0.4pt} 
\linespread{1.3}

\usepackage{titlesec}
\titleformat{\chapter}[display]{\normalfont\Huge\bfseries}{\chaptertitlename\ \thechapter}{20pt}{\Huge}
\titleformat{\section}{\LARGE\bfseries}{\thesection}{1em}{}


\def\N{\mathbb{N}}
\def\R{\mathbb{R}}
\def\n{\noindent}
\def\s{\vspace{10pt}}


\theoremstyle{definition}
\newtheorem{definicija}{Definicija}

\theoremstyle{plain}
\newtheorem{izrek}{Izrek}

\theoremstyle{plain}
\newtheorem{trditev}{Trditev}

\theoremstyle{plain}
\newtheorem{posledica}{Posledica}

\theoremstyle{remark}
\newtheorem*{opomba}{Opomba}

\usepackage{thmtools}
\declaretheoremstyle[
    spaceabove=10pt,
    spacebelow=10pt,
    bodyfont=\normalfont,
    headfont=\bfseries,
    postheadspace=0.5em,
    qed=$\lozenge$ 
]{example}
\declaretheorem[style=example, unnumbered]{zgled}


\usepackage{filemod}
\title{\Huge Verjetnost 1}
\author{Napisal: Jon Pascal Miklavčič}
\date{\filemodprintdate{\jobname}}


\begin{document}

\frontmatter

\maketitle

\chapter*{Predgovor}
Zapiski v tej skripti so bili osnovanih na rokopisih predavanj iz študijskih let pred letom 2023 in doknčno dopolnjeni z zapiski iz predavanj iz leta 2024.

\tableofcontents

\mainmatter

\chapter{Neformalni uvod v verjetnost}

Začetki verjetnosti so v 17. stoletju, iz iger na srečo (kartanje, kockanje, \dots): 
\begin{itemize}
    \item 17. stol.: Fermant, Pascal, Bernulli;
    \item 18./19. stol.: Laplace, Poisson, Čebišev, Markov;
    \item 20. stol.: Kolmogorov.
\end{itemize}

\n Izvajamo poskus in opazujemo določen pojav, ki ga imenujemo \emph{dogodek}. Ta se lahko zgodi ali ne. 

\begin{zgled}
    Poskus je met kocke. Da pade šestica, da pade sodo število pik pa sta dogodka.
\end{zgled}

\n Poskus ponovimo $n$-krat. Opazujemo dogodek $A$. S $k_n(A)$ označimo \emph{frekvenco dogodka} $A$, t.j. število tistih ponovitev poskusa, pri katerih se je dogodek $A$ zgodil. Naj bo $f_n(A)=\frac{k_n(A)}{n}$ \emph{relativna frekvenca} dogodka $A$. Dokazati je mogoče, da zporedje $\left\{f_n(A)\right\}_n$ konvergira k nekemu številu $p \in [0, 1]$; $f_n(A) \xrightarrow{n \to \infty} p$. Dobimo:

\textbf{Statistično definicijo verjetnosti:}

$$P(A):= p $$

\n Pogosto lahko verjetnost določimo vnaprej in sicer s: 

\textbf{Klasično definicijo verjetnosti:}

$$P(A):= \frac{\# \text{ ugodnih izidov za dogodek } A}{\# \text{ vseh izidov}}$$

pri pogoju, da imajo vsi izidi \emph{enake} možnosti.

\begin{zgled}
    Met poštene kocke: 
    $$P(\text{sodo število pik}) = \frac{3}{6} = \frac{1}{2}$$
\end{zgled}

\begin{zgled}
    Kolikšna je verjetnost, da pri metu dveh poštenih kock znaša vsota pik $7$?
    
    \n Možne vsote so: $2, 3, 4, \dots , 12$. Opazimo, da je vseh vsot $11$ in od tega $1$ ugodna. Ali to pomeni, da $P(A)=\frac{1}{11}$. Ne! Izidi niso enkaoverjetni. 
    
    Na primer $2$ lahko dobimo samo kot $2=1+1$, $5$ pa kot $5=2+3=1+4=4+1=3+2$. 

    Torej vsi možni izidi, bodo urejeni pari $(x, y)$, kjer $x, y \in[1,6] \subset \mathbb{N}$
    $$
    \begin{array}{cccc}
        (1,1) & (1,2) & \cdots & (1,6) \\
        (2,1) & (2,2) & \ddots & (2,6)\\
        \vdots & \ddots & \ddots & \vdots \\
        (6,1) & (6,2) & \cdots & (6,6) \\
    \end{array}
    $$

    Vseh izidov je torej $36$ in od tega je $6$ ugodnih. Torej $P(A)= \frac{6}{36}=\frac{1}{6}$.
\end{zgled}

\n Če je izidov neskončno, si lahko pomagamo s \textbf{Geometrijsko definicijo verjetnosti.}

\begin{zgled}
    Osebi se dogovorita za srečanje med 10. in 11. uro. Čas prihoda je slučajen. Vsak od njiju po prihodu čaka največ 20 minut. Če v tem času drugega ni, odide. Najdlje čaka do 11. ure. Kolišna je vrejetnost srečanja?

    Čas začnemo šteti ob 10. uri. Vsi izidi so urejeni pari $(x, y) \in [0,1] \times[0,1]$. Ugodni izidi so $|x-y| \leq \frac{1}{3}$. Torej: 
    $$
    \begin{aligned}
        \text{1) } x \geq y &: x-\frac{1}{3} \leq y \\
        \text{2) } x \leq y &: y-x \leq \frac{1}{3} \iff y \leq x+\frac{1}{3}
    \end{aligned}
    $$

    Torej je $$P(\text{srečanja})=\frac{1-\left(\frac{2}{3}\right)^2}{1}=\frac{5}{9}$$
\end{zgled}

\n Teorija mere se ukvarja z splošnim zapisom geometrijske definicije. 

\begin{zgled}
    Vzamemo $m, n \in \mathbb{N}$, $m>n$. $n$ krogljic slučajno razporedimo v $m$ posod. Kolikšna je verjetnost dogodka, da so vse krogljice v prvih $n$ posodah, v vsaki ena?

    To je pomankljivo zastavljena naloga. Ne vemo namreč, ali med seboj krogljice razlikujemo, ali ne. Za dodatno predpostavko se ponujajo 3 možnosti: 

    1) \textbf{krogljice razlikujemo:}
    
    \n Število vseh izidov v tem primeru je ravno število \emph{variacij} $m$ elementov na $n$ mestih \emph{s ponavljanjem}. Za vsako od $n$-tih kroglic imamo $m$ možnosti, torej je vseh možnosti $m \cdot m \cdots m=m^n$.
    
    \n Število ugodnih izidov pa je ravno število \emph{permutacij} $n$ krogljic v prvih $n$ posodah. Torej je ugodnih možnosti $n(n-1) \ldots 2 \cdot 1=n!$.
    
    Torej je $$P(A)=\frac{n !}{m^n}$$

    2) \textbf{krogljic ne razlikujemo:} 
    
    \n V vsaki posodi je lahko več krogljic. Število vseh izidov je ravno število \emph{kombinacij s ponavljanjem}. Število kombinacij $m$ elementov s ponavljanjem na $n$ mestih je: 
    $$
    \binom{n+m-1}{n} = \binom{n+m-1}{m-1}
    $$

    \n Postavimo $n$ krogljic in med njih razporedimo $m-1$ črtic, ki predstavljajo stene posod:
    $$
    |\underbrace{\circ|\circ|\circ\circ |\circ\circ  \cdots |\circ |}_{n \text{ krogljic, } m-1 \text{ črtic}}|
    $$

    \n Na $n+m-1$ mestih moramo določiti $n$ krogljic. Ugoden izid je samo eden: 
    $$
    |\circ|\circ|\circ|\circ \cdots \circ|||| \cdots |
    $$

    Torej je $$P(A)=\frac{1}{\binom{n+m-1}{n}}$$

    3) \textbf{krogljic ne razlikujemo, v vsaki posodi je kvečjemu ena krogljica:} 
    
    Število vseh izidov je ravno število \emph{kombinacij brez ponavljanja} $\binom{m}{n}$. Ugoden izid je eden. 

    Torej je $$P(A)=\frac{1}{\binom{m}{n}}$$
\end{zgled}

\begin{opomba}
V fiziki so krogljice delici (atomi, molekule, ...), posode pa fazna stanja, v katerih so lahko delci. Glede na zgornje primere ločimo:

\begin{enumerate}
    \item Maxwell-Boltzmannovo statistika, ki velja za molekule plina.
    \item Bose-Einsteinovo statistika, ki velja za delce imenovane bozoni.
    \item Fermi-Diracovo statistika, ki velja za fermione. 
\end{enumerate}

\n Diracovo izključitveno načelo.

\end{opomba}

\newpage

\chapter{Aksiomatična definicija verjetnosti}

Imamo prostor vseh izidov oz. \emph{vzorčni prostor} $\Omega$ (možna oznaka je tudi $\mathcal{G}$). Dogodki so nekatere (ne nujno vse) podmnozice $\Omega$.

\begin{zgled}
    Met kocke. Vzorčni prostor je $\Omega=\{1,2,3,4,5,6\}$, dogodki pa so poljubne podmnožice $\Omega$, to je $\mathcal{P}(\Omega) = 2^{\Omega}$. Na primer $A = \{2,4,6\}$ je dogodek, da pade sodo število pik. 
\end{zgled}

\n Računanje z dogodki:

\begin{enumerate}
    \item \emph{Vsota dogodkov} oz. \emph{unija dogodkov} (zgodi se vsaj enden od dogodkov): $$A + B = A \cup B$$
    \item \emph{Produkt dogodkov} oz. \emph{presek dogodkov} (zgodita se oba dogodka hkrati): $$A \cdot B = A \cap B$$
    \item \emph{Nasprotni dogodek} oz. \emph{komplement dogodka} (dogodek se ne zgodi): $$\bar A = A^c$$
\end{enumerate}

\n Pravila za računanje z dogodki: 

\begin{enumerate}
    \item \textbf{idempotentnost}: $A \cup A=A=A \cap A$
    \item \textbf{komutativnost}: $A \cup B=B \cup A, \quad A \cap B=B \cap A$
    \item \textbf{asociativnost}: $$\begin{aligned} & (A \cup B) \cup C=A \cup(B \cup C) \\ & (A \cap B) \cap C=A \cap(B \cap C)\end{aligned}$$
    \item \textbf{distributivnost}: $$\begin{aligned} & (A \cup B) \cap C=(A \cap C) \cup(B \cap C) \\ & (A \cap B) \cup C=(A \cup C) \cap(B \cup C)\end{aligned}$$
    \item \textbf{de Morganova zakona}: $$\begin{aligned} & (A \cap B)^c=A^c \cup B^c \\ & (A \cup B)^c=A^c \cap B^c\end{aligned}$$ \\ Še več: $$ \left(\bigcap_i A_i\right)^c=\bigcup_i A_i^c, \quad \left(\bigcup_i A_i\right)^c=\bigcap_i A_i^c$$
\end{enumerate}

\n V splošnem ni vsaka podnožica množice $\Omega$ dogodek. Neprazna družina podmnožic (dogodkov) $\mathcal{F}$ v $\Omega$ je \emph{$\sigma$-algerba}, če zanjo velja: 

\begin{enumerate}
    \item $\Omega \in \mathcal{F}$
    \item $A \in \mathcal{F} \implies A^c \in \mathcal{F}$
    \item $A_1, A_2, \ldots \in \mathcal{F} \implies \bigcup_{i=1}^{\infty} A_i \in \mathcal{F}$
\end{enumerate}

\n Elementi v $\mathcal{F}$ so dogodki. Če v točki 3. zahtevamo manj: 

\begin{enumerate}[start=3,label={\arabic*.*}]
    \item $A, B \in \mathcal{F} \implies A \cup B \in \mathcal{F}$
\end{enumerate}

potem pravimo, da je $\mathcal{F}$ \emph{algebra}. \s

V algebri imamo potem tudi zaprtost za končne unije: $A_1, A_2, \ldots, A_n \in \mathcal{F} \implies A_1 \cup A_2 \cup \cdots \cup A_n \in \mathcal{F}$. Ker po de-Morganu velja $\bigcap_i A_i=\left(\bigcup_i A_i^c\right)^c$, je algebra zaprta za končne preseke, $\sigma$-algebra pa celo za števne preseke. Ker velja $A \setminus B=A \cap B^c$, je algebra zaprta za razlike. 

Vsaka algebra vsebuje $\{\emptyset, \Omega\}$. Ker je $\mathcal{F}$ neprazna, obstaja $A \in \mathcal{F}$ in zato tudi $\Omega=A \cup A^c \in \mathcal{F}$ in $\emptyset=\Omega^c \in \mathcal{F}$. Tako dobimo, da je $\{\emptyset, \Omega\}$ najmanšja možna ($\sigma$-)algebra, $\mathcal{P}(\Omega)$ pa največja možna ($\sigma$-)algebra.

\begin{zgled}
    Za $A \neq \emptyset \neq \Omega$ je najmanjša ($\sigma$-)algebra, ki vsebuje $A$ enaka $\left\{\emptyset, A, A^c, \Omega\right\}$. 

    \n Za $\Omega=\{1,2,3\}$ in $A=\{1,2\}$, je potem taka $\sigma$-algebra $\{\emptyset,\{3\},\{1,2\},\{1,2,3\}\}$.
\end{zgled}

Dogodka $A$ in $B$ sta \emph{disjunkta} oz. \emph{nezdružljiva} če je $A \cap B=\emptyset$. 

Zaporedje $\{A_i\}_i$ (končno ali števno mnogo) je \emph{popoln sistem dogodkov}, če velja: $$\bigcup_i A_i=\Omega \quad \text{in} \quad A_i \cap A_j = \emptyset \quad \text{za} \quad i \neq j.$$ 

\n Naj bo $\mathcal{F}$ $\sigma$-algebra na $\Omega$. \textbf{Verjetnost} na $(\Omega, \mathcal{F})$ je preslikava $P:\mathcal{F} \to \mathbb{R}$ z lastnostmi:

\begin{enumerate}
    \item Za vsak $A \in \mathcal{F}$: $P(A) \geq 0 $
    \item $P(\Omega) = 1$
    \item Za poljubne paroma nezdružljive dogodke $\{A_i\}_{i=1}^{\infty}$ velja števna aditivnost: $$P\left(\bigcup_{j=1}^{\infty} A_i\right)=\sum_{i=1}^{\infty} P\left(A_i\right)$$
\end{enumerate}

\n Lastnosti verjetnosti $P$: 

\begin{enumerate}[label=(\alph*)]
    \item $P(\emptyset) = 0$. 
    \begin{proof}
        V lastnosti 3. vzamamo $A_i = \emptyset$ za vsak $i$: 
        $$
        P\left(\bigcup_i \emptyset\right)=P(\emptyset)+P(\emptyset)+P(\emptyset)+\cdots = 0+0+0+ \cdots = 0
        $$
    \end{proof}
    \item $P$ je \emph{končno aditivna}, t.j. za končno mnogo paroma nezdružljivih dogodkov $\{A_i\}_{i=1}^{n}$ velja: 
    $$
    P\left(A_1 \cup \cdots \cup A_n\right)=P\left(A_1\right)+P\left(A_2\right)+\cdots+P\left(A_n\right)
    $$
    \begin{proof}
        V lastnosti 3. vzamemo $A_{n+1}=A_{n+2}=\cdots=\emptyset$ in upoštevamo lastnost (a). 
    \end{proof}
    \item $P$ je \emph{monotona}, t.j. velja: 
    $$
    A \subseteq B \implies P(A) \leq P(B)
    $$ 
    Še več: iz $A \subseteq B$  sledi $P(B \setminus A)=P(B)-P(A)$.
    \begin{proof}
        Ker je $B=A \cup(B \setminus A)$ in $A \cap(B \setminus A)=\emptyset$ je $P(B)=P(A)+P(B \setminus A)$ zaradi lastnosti (b).
    \end{proof}
    \item $P(A^c) = 1-P(A)$
    \begin{proof}
        V (c) vzamemo $B = \Omega$.
    \end{proof}
    \item $P$ je \emph{zvezna}, t.j.:
    
    \begin{enumerate}[label=(\roman*)]
        \item $$ A_1 \subseteq A_2 \subseteq A_3 \subseteq \cdots \implies P\left(\bigcup_{i=1}^{\infty} A_i\right)=\lim _{n \to \infty} P\left(A_n\right)$$
        \item $$ B_1 \supseteq B_2 \supseteq B_3 \supseteq \cdots \implies P\left(\bigcap_{i=1}^{\infty} B_i\right)=\lim _{n \to \infty} P\left(B_n\right)$$
    \end{enumerate}

    \begin{proof}
        \begin{enumerate}[label=(\roman*)]
            \item Definiramo $C_1 = A$ in $C_i = A_i \setminus A_{i-1}$ za $i = 2, 3, \ldots$. Potem je $A_n=C_1 \cup \ldots \cup C_n$, kjer velja $C_i \cap C_j=\emptyset$ za $i \neq j$ in $\bigcup_{i=1}^{\infty} A_i=\bigcup_{i=1}^{\infty} C_i$. Torej je: 
            $$ \begin{aligned} P\left(\bigcup_{i=1}^{\infty} A_i \right) &= P\left(\bigcup_{i=1}^{\infty} C_i\right) \\ &= \sum_{i=1}^{\infty} P(C_i) \\ &= \lim _{n \to \infty} \sum_{i=1}^{n} P(C_i) \\ &= \lim _{n \to \infty} P\left(\bigcup_{i=1}^n C_i\right) \\ &= \lim_{n \to \infty} P(A_n) \end{aligned}$$ 
            \item Ker $B_1 \supseteq B_2 \supseteq B_3 \supseteq \cdots$, je potem $B_1^c \subseteq B_2^c \subseteq B_3^c \subseteq \cdots$. Po (i) potem velja 
            $$P\left(\bigcup_{i=1} B_i^c\right)=\lim _{i \to \infty} P\left(B_i^c\right)$$ 
            Toda $$ \bigcup_{i=1}^{\infty} B_i^c=\left(\bigcap_{i=1}^{\infty} B_i\right)^c \implies 1-P\left(\bigcap_{i=1}^{\infty} B_i\right)=\lim_{i \to \infty}\left(1-P(B_i\right)) $$ Od koder sledi željena enakost. 
        \end{enumerate}
    \end{proof}
\end{enumerate}

\textbf{Verjetnostni prostor} je trojica $(\Omega, \mathcal{F}, P)$

\begin{zgled}[Končni ali števni verjetnostni prostor]
    $\Omega=\left\{\omega_2, \omega_2, \omega_3, \ldots\right\}$ končno ali števno mnogo izidov. $\left\{\omega_1\right\},\left\{\omega_2\right\},\left\{\omega_3\right\}, \ldots$ je popoln sistem dogodkov, neka podmnožica v $\Omega$ je končna ali števna unija teh dogodkov. Torej $\mathcal{F}=\mathcal{P}(\Omega)$. Vzamemo:
    $$
    A=\bigcup_{i: \omega_i \in A}\left\{\omega_0\right\} 
    $$
    Če označimo $P(\{\omega_i\}) = p_i \geq 0$ je $\sum_i p_i=1$ in $P(A)=\sum_{i: \omega_i \in A} p_i$, $A \subseteq \Omega$.

    Če ima $\Omega$ $n$ elementov in $p_i = \frac{1}{n}$ za $i=1, 2, \ldots, n$. Potem je $P(A)=\frac{|A|}{n}=\frac{\operatorname{moč}(A)}{n}$. To je klasična definicija verjetnosti. 
\end{zgled}

\begin{zgled}[Neskončni neštevni verjetnostni prostor]
    Primer srečanja dveh oseb, kjer $\Omega=[0,1] \times[0,1]$. Za $\sigma$-algebro $\mathcal{F}$ ne moremo vzeti vseh podmnožic, radi pa bi jih vzeli čim več.

    $\mathcal{F}$ naj bo najmanjša $\sigma$-algebra, ki vsebje vse odprte pravokotnike $(a, b)\times(c, d)$ (izkaže se, da je isto, če vzamemo zaprte pravokotnike). $\mathcal{F}$ imenujemo \emph{Borelova $\sigma$-algebra}.

    Verjetnost definiramo na pravokotnikih kot: 
    $$
    P((a, b) \times(c, d))=(b-a)(d-c)
    $$
    Ni lahko videti, da lahko $P$ razširimo do verjetnosti na $\mathcal{F}$. $P$ pa ne moremo razširiti na $\mathcal{P}(\Omega)$. Problem je števna aditivnost. 

    To je geometrijska definicija verjetnosti.
\end{zgled}

\chapter{Pogojna verjetnost}



\end{document}